\chapter{Revolución Mexicana}

Charles Rosher, un camarógrafo inglés en el naciente cine mudo de Estados Unidos cuenta su corta experiencia en las cárceles mexicanas después de haber documentado las batalla de Ojinaga al norte del país, en la que fue ''arrestado y encarcelado por los soldados federales, hasta que me llevaron frente al general Castro. Se dio cuenta del botón masón que llevaba en el saco y me hizo el saludo masón ¡Él era masón también! Entonces me liberaron.''\cite{rocha2003}

Esta historia, sin embargo, pronto llegó a ser descubierta como falsa. Rosher en realidad tomó la anécdota del camarógrafo independiente Charles Pryor, dueño de la compañía fílmica \emph{El Paso FeaturevFilm Company}\cite{worthington2010}, quien sí fue arrestado, además de que existe un video que lo muestra frente a la cárcel de Ojinaga, en la que se le ve el botón masónico en la solapa.